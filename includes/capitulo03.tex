%Empieza configuracion de capitulo
\setstretch{1.0}
\titleformat{\chapter}[block]{\Large\bfseries}{CAP'ITULO \Huge\thechapter\vspace{25 pt}}{0 pt}{\\\fontsize{26}{36}\selectfont}
\titlespacing{\chapter}{0 pt}{30 pt}{50 pt}[0 pt]
\titleformat{\section}{\Large\bfseries}{\thesection}{0 pt}{\hspace{30 pt}}
\titleformat{\subsection}{\large\bfseries}{\thesubsection}{0 pt}{\hspace{30 pt}}
\pagestyle{fancy}
\fancyhead[LO,LE]{\footnotesize\textit{\leftmark}}
\fancyhead[RO,RE]{\thepage}
\fancyfoot[CO,CE]{}
%Termina configuracion de capitulo

\chapter{Planeaci'on} %Cambia al nombre de tu capitulo
\setstretch{1.5} %Regresa el interlineado a 1.5

\normalsize

\section{Cronograma}
\noindent
El cronograma puede representarse en forma gr'afica, pero conviene incluir una parte en la que se justifique la selecci'on de actividades. Procure utilizar t'ecnicas de administraci'on de proyectos (diagramas de Gantt, por ejemplo), para generar un cronograma lo m'as eficiente posible. Recuerde que el tiempo es escaso y no se debe desperdiciar.

\section{Presupuesto}
\noindent
Es muy probable que, para la realizaci'on de este proyecto, usted s'olo cuente con sus propios fondos, a menos que haya logrado involucrar a su empresa o alguna otra organizaci'on. De todas formas, plantear un presupuesto representa una disciplina valiosa que le puede ahorrar recursos y dolores de cabeza, por lo que se recomienda sea cuidadoso y considere todos los posibles costos de manera exhaustiva.

\section{Difusi'on}
\noindent
Esta secci'on es opcional, pero puede ser 'util mencionar en qu'e congresos se piensa presentar los resultados de la investigaci'on a realizar en el proyecto, as'i como las revistas en las que se podr'ian publicar los art'iculos correspondientes.
\newpage
En muy raros casos, alg'un cap'itulo de tesis no requerir'a el uso de referencias bibliogr'aficas (por ejemplo, el cap'itulo de conclusiones o de trabajo futuro). En esos casos, usted no tendr'a que incluir referencias al final del cap'itulo, que es justamente lo que se hace en el presente caso.

\clearpage