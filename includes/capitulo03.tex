%Empieza configuracion de capitulo
\setstretch{1.0}
\titleformat{\chapter}[block]{\Large\bfseries}{CHAPTER \Huge\thechapter\vspace{25 pt}}{0 pt}{\\\fontsize{26}{36}\selectfont}
\titlespacing{\chapter}{0 pt}{30 pt}{50 pt}[0 pt]
\titleformat{\section}{\Large\bfseries}{\thesection}{0 pt}{\hspace{30 pt}}
\titleformat{\subsection}{\large\bfseries}{\thesubsection}{0 pt}{\hspace{30 pt}}
\pagestyle{fancy}
\fancyhead[LO,LE]{\footnotesize\emph{\leftmark}}
\fancyhead[RO,RE]{\thepage}
\fancyfoot[CO,CE]{}
%Termina configuracion de capitulo

\chapter{Theoretical Framework}
\setstretch{1.5} %Regresa el interlineado a 1.5

\normalsize
\noindent

This is a new topic that will require quite a few topics to fully understand
further experiments and why we decided to do them. We will try to cover as much
of the topics needed to fully understand the problem and the proposed
solution. However will not cover deep topics as Operating Systems architecture
nor parallel programming concepts.


\section{The need of parallel computing}
\noindent

Computer technology has made incredible progress in the roughly 60 years since
the first general-purpose electronic computer was created. Today, less than 500
USD will purchase a personal computer that has more performance, more main
memory, and more disk storage than a computer bought in 1985 for 1 million
dollars. This rapid improvement has come both from advances in the technology
used to build computers and from innovation in computer design. \cite{Hennessy}

As we have seen it was around the years 2003 to 2005 that a dramatic change
seized the semiconductor industry and the manufactures of processors. The
increasing of computing performance in processors, based on simply screwing up
the clock frequency, could not longer be holded. Scaling of the technology
processes, leading to smaller channel lengths and shorter switching times in
the devices, and measures like instruction-level-parallelism and out-of-order
processing, leading to high fill rates in the processor pipelines, were the
guarantors to meet Moore’s law.\cite{AMDAHL}

The answer of the industry to that development, in order to still meet Moore’s
law, was the shifting to real parallelism by doubling the number of processors
on one chip die. This was the birth of the multi-core area. The benefits of
multi-core computing, to meet Moore’s law and to limit the power density at the
same time, at least at the moment this statement holds, are also the reason
that parallel computing based on multi-core processors is underway to capture more
and more applications, the world of embedded processing for example.\cite{MATTSON}

Where do we find these task parallelism in embedded systems? A good example are
automotive applications Multi-core technology in combination with a broadband
efficient network system offers the possibility to save components, too, by
migrating functionality that is now distributed among a quite large number of
compute devices to fewer cores.


\section{Distributed Systems}
\noindent

The need of more complex and smart applications has risen the bar to create
distributed systems based on parallel embedded platforms. By definition: A
distributed system consists of a collection of autonomous computers, connected
through a network and distribution middleware, which enables computers to
coordinate their activities and to share the resources of the system, so that
users perceive the system as a single, integrated computing facility.

Advantages:

\begin{enumerate}

\item \textbf{Partitioning Workload}:

By partitioning the workload onto multiple processors, each processor is now
responsible for only a fraction of the workload. The processors can now afford
to slow down by dynamic voltage scaling (DVS) to run at more power-efficient
states, and the degraded performance on each processor can still contribute to
an increased system-wide performance by the increased parallelism.

\item \textbf{Heterogeneous HW}:
Another advantage with a distributed scheme is that heterogeneous hardware such
as DSP and other accelerators can further improve power efficiency of various
stages of the computation through specialization.

\end{enumerate}


Disadvantages:

\begin{enumerate}
\item \textbf{Network}: Despite the fact the distributed systems may have
many attractive properties, they pay a higher price for message-passing
communications. Each node now must handle not only communication
with the external world, but also extra communication on the internal network.
As a result, even if the actual data payload is not large
on an absolute scale, the communication appears very expensive and
does not scale to a few more nodes

\item \textbf{Lack of optimized OS}: A typical embedded system often does
not contain an operating system. Crafting distributed programs on such a
bare-bone platform is extremely difficult and error-prone. Although many
higher-level abstractions such as Message Passing Interfaces (MPI)
have been proposed to facilitate distributed programming, these abstraction
layers require extensive system resources with comprehensive operating systems
support, which may not be available to an embedded platform
\end{enumerate}

We will try to fulfill these minimal requirements described by (Tanaka) in our
system:

\begin{itemize}
\item  Multiple autonomous components
\item Components are not shared by all users
\item Software runs in concurrent processes on different processors
\item Resource Sharing : Ability to use any hardware, software or data anywhere in
the system
\item Openness: Differences in data representation of interface types on different
processors (of different vendors) have to be resolved
\item Concurrency : Components in distributed systems are executed in concurrent
processes
\item Scalability: Components should not need to be changed when scale of a system
increases
\item Fault Tolerance: Distributed systems must maintain availability even at low
levels of hardware/software/network reliability.
\item Transparency: Distributed systems should be perceived by users and
application programmers as a whole rather than as a collection of cooperating components.
\end{itemize}

\section{Embedded Linux systems}

Computers are everywhere , this fact, of course, is not a surprise to anyone 
who hasn't been living in a cave during the past 25 years or so. And you
probably know that computers aren't just on our desktops, in our kitchens, and, 
increasingly, in our living rooms holding our music collections. They're also 
in our microwave ovens, our regular ovens, our cellphones, and our portable 
digital music players.

Until not too long ago, embedded systems were not very powerful, and they ran
special-purpose, proprietary operating systems that were very different from
industry-standard ones. (Plus, they were much harder to develop for.) Today,
embedded computers are as powerful as, if not more than, a modern home
computer. (Consider the high-end gaming consoles, for example.)

Along with this power comes the capability to run a full-fledged operating
system such as Linux. Using a system such as Linux for an embedded product
makes a lot of sense. A large community of developers are making it possible.
The development environment and the deployment environment can be surprisingly
similar, which makes your life as a developer much easier. And you have both
the security of a protected address space that a virtual memory-based system
gives you, and the power and flexibility of a multiuser, multiprocess system.
That's a good deal all around. For this reason, companies all over the world
are using Linux on many devices

In the beginning embedded systems were most commonly resource constrained
compared to the typical desktop PC. Embedded systems often had limited memory,
small or no hard drives, and sometimes no external network connectivity.
Frequently, the only user interface was a serial port and some LED's. But today
there has been a change in the last point: external network connectivity.

The internet of things is the evolution of the embedded world in terms of
connectivity. The IoT systems keep being resource constrained but with the
capability to be connected to an internet network where to send the data to
external servers (where it will be analyzed,stored and presented in a web base
user interface to the end user)

In this new world the advantages (flexibility and scalability) of using Linux
as the core of the OS Embedded Distributions turn out to make a nightmare for
the embedded developers with too many options and no standard work to re use.
In the end the developer ended with a "Frankenstein" embedded Linux
distribution with very low maintainability and robustness. In 2010 there was a
change in the industry of embedded systems, the announce of a project to solve
these kind of problems: The Yocto project.

\section{The Yocto project}
\noindent
The Yocto Project is an open source collaboration project that provides
templates, tools and methods to help you create custom Linux-based systems for
embedded products regardless of the hardware architecture. It was founded in
2010 as a collaboration among many hardware manufacturers, open-source
operating systems vendors, and electronics companies to bring some order to the
chaos of embedded Linux development.

As an open source project, the Yocto Project operates with a hierarchical
governance structure based on meritocracy and managed by its chief architect,
Richard Purdie, a Linux Foundation fellow. This enables the project to remain
independent of any one of its member organizations, who participate in various
ways and provide resources to the project.

Why use the Yocto Project? It's a complete embedded Linux development
environment with tools, meta-data, and documentation. The
free tools are easy to get started with, powerful to work with (including
emulation environments, debuggers, an Application Toolkit Generator, etc.) and
they allow projects to be carried forward over time without causing you to lose
optimizations and investments made during the project’s prototype phase. The
Yocto Project fosters community adoption of this open source technology
allowing its users to focus on their specific product features and development.

The Yocto Project through the Poky build tool provides an open source
development environment (Figure XXX) targeting the ARM, MIPS, Power-PC and x86 
architectures for a variety of platforms including x86-64 and emulated ones. 
You can use components from the the Yocto Project to design, develop, build, 
debug, simulate, and test the complete software stack using Linux, the X Window
System, GNOME Mobile-based application frameworks, and Qt frameworks. 

For complete information on the Yocto Project, you should check out the Yocto
Project Website. You can find the latest builds, breaking news, full
development documentation, and a rich Yocto Project Development Community into
which you can tap.


\section{Power and Performance}

\noindent
When we say one computer is faster than another is, what do we mean? The user
of a desktop computer may say a computer is faster when a program runs in less
time, while an Amazon.com administrator may say a computer is faster when it
completes more transactions per hour. The computer user is interested in 
reducing response time (the time between the start and the completion of an 
event) also referred to as execution time. The administrator of a large data 
processing center may be interested in increasing throughput (the total amount 
of work done in a given time.) In comparing design alternatives, we often want
to relate the performance of two different computers, say, X and Y. The phrase 
“X is faster than Y” is used here to mean that the response time or execution 
time is lower on X than on Y for the given task. In particular, “X is n times 
faster than Y” will mean (Xizhou Feng)



\clearpage
