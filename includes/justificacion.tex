%Empieza configuracion
\setstretch{1.0}
\titleformat{\chapter}{\Huge\bfseries}{\thechapter}{0 pt}{\rule{340 pt}{3 pt}\\}
\titlespacing{\chapter}{100 pt}{-25 pt}{40 pt}[10 pt]	
\pagestyle{fancy}
\fancyhead[RO,RE]{\thepage}
\fancyfoot[CO,CE]{}
%Termina configuracion

\chapter*{Justification}
\addcontentsline{toc}{chapter}{Justification}
\setstretch{1.5} %Regresa el interlineado a 1.5

\normalsize
\noindent

One could think that we are overreacting to the fact that the rise of IoT will
lead to an explosion in the volume of data collected, transmitted and
processed, but actually we are already seeing this problem in current industry.
Such is the case of the Aircraft industry. On March of 2013 Virgin Atlantic was
preparing for a significant increase in data as it embraces the internet of
things, with a new fleet of highly connected planes each one of these modern
planes is expected to create over half a terabyte of data per flight.

In particular, the Boeing 787 aircraft ordered by Virgin Atlantic for delivery
in 2014  dramatically increase the volume of data the airline will need to deal
with (half terabyte in a transatlatic flight).  Because they can't handle that
much of data everyday comming from various airplanes they are looking
for cloud base solutions inside the airplanes. This is an expensive solution
since the cost in space and energy required to handle a cloud base solution
inside an airplane are high.

But it's not just about the amount of data collected, transmited and processed is
also about the energy efficient handling. \cite{Bergelt}. The development of
wireless sensor networks has reached a point where each individual node of a
network may store and deliver a massive amount of (sensor-based) information at
once or over time.  Right now the total amount of user data (data payload) to
be stored or processed doubles every two years. Consequently, data will become
a problem for traditional data aggregation strategies traffic-wise as well as
with regard to energy efficiency. 

As we can see the lack of standard solutions for self sustainable networks is
creating real problems among the industry. This gave us the justification to
generate a comparison of how many embedded systems might be necesary to have
the same data processing capabilities than in a centralized system. Hopefully
with this information future projects might consider the option use the
embedded systems they have to manage their own processing tasks.

\clearpage
