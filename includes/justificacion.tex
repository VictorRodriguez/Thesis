%Empieza configuracion
\setstretch{1.0}
\titleformat{\chapter}{\Huge\bfseries}{\thechapter}{0 pt}{\rule{340 pt}{3 pt}\\}
\titlespacing{\chapter}{100 pt}{-25 pt}{40 pt}[10 pt]	
\pagestyle{fancy}
\fancyhead[RO,RE]{\thepage}
\fancyfoot[CO,CE]{}
%Termina configuracion

\chapter*{Justification}
\addcontentsline{toc}{chapter}{Justification}
\setstretch{1.5} %Regresa el interlineado a 1.5

\normalsize
\noindent

Nowadays there are multiple examples of IoT systems with enough processing
capabilities to avoid the cost of send data to a centralized processing system.
Despite the variety of IoT applications, which goes from greenhouse monitoring
systems \cite{Liu-Dan} to safety management systems for oil depots \cite{Du}, all
of them are based on the same IoT principle of send data over the internet to a
centralized processing system without using the processing capabilities of the
IoT device 

Taking the greenhouse example of \cite{Du}, where the system realizes the
remote intelligent control of the environment through the internet. Is easy to
see how applying the IoT technology is possible to improve the operation and
system flexibility by using the wireless sensor network instead of the
traditional wired network. Nevertheless, it uses a dedicated microprocessor
just to transmit data to the centralized control system; meanwhile, the
intelligence in this centralized control system is just a simple conditional
rule.

However, there are other projects where they  use the full compute power of the
ultra-low power microprocessors. Such is the case of \cite{Wun} where they use
an embedded system as the computing platform of an autonomous industrial tank
floor inspection robot. In this project, the Microprocessor is running the
Robot Operating System (ROS) for tank floor mapping and navigation of the
robot. The reason why an embedded platform was selected is due to his low power
consumption and enough processing capabilities. 

But projects where the embedded system became the central processing unit are
not common. This is generating an excessive use of resources for transmission
and process of data.  If the trend continues the rise of IoT will lead to an
explosion in the volume of data collected, transmitted and processed. According
to \cite{Beraelt}, the total amount of user data (data payload) to be stored or
processed doubles every two years.The first problems due to this trend  are
currently being faced by industry; such is the case of the aircraft industry

In 2013, the Virgin Atlantic \cite{Virgin} airline announced that the Boeing 787
aircraft was going to be one of the first highly connected planes. Each one of
these planes was expected to create over half a terabyte of data per
transatlantic flight.  Due to the costs of handle and transmit that much data
every day, the airline started to look for cloud base solutions inside the
airplanes. This is an expensive solution since the cost in space and energy
required to handle a cloud base solution inside an airplane are high.

As we can see the current IoT architectures are not efficiently using the
compute resources they have. In addition to this the lack of standard
methodologies to create self sustainable (in terms of data processing) IoT
networks is creating real problems among the industry.  This gave us the
justification to generate a comparison of how many low power compute systems
might be necessary to have the same data processing capabilities than in an
external computer. Hopefully, with this information future projects might
consider the option of do not send data over the internet, reducing the data
payload.

\clearpage
