%Empieza configuracion
\setstretch{1.0}
\titleformat{\chapter}{\Huge\bfseries}{\thechapter}{0 pt}{\rule{340 pt}{3 pt}\\}
\titlespacing{\chapter}{100 pt}{-25 pt}{40 pt}[10 pt]	
\pagestyle{fancy}
\fancyhead[RO,RE]{\thepage}
\fancyfoot[CO,CE]{}
%Termina configuracion

\chapter*{Justification}
\addcontentsline{toc}{chapter}{Justification}
\setstretch{1.5} %Regresa el interlineado a 1.5

\normalsize
\noindent

Nowadays, there are multiple examples of IoT systems with large processing
capabilities, which could be exploited to avoid sending data to a centralized
processing system. Despite of this characteristic, it is frequently seen that
there is a large variety of IoT applications that continue sending data over
the Internet to a centralized processing system, without using the processing
capabilities of IoT devices.

There are several IoT examples in which the system performs remote control of
the environment through the Internet. In all of them it can be seen that
applying IoT technology it is possible to improve the operation and system
flexibility, by using the wireless sensor network instead of the traditional
wired sensor network. In several cases, it is observed that such systems
dedicate a general-purpose microprocessor just to transmit data to the
centralized control system; even though, the problem being solved in such type
of centralized control systems is a simple conditional rule. However, there are
other projects that use ultra-low power microprocessors.  There are several
industrial applications that use embedded systems for autonomous robots. In
such scenarios embedded platforms are chosen due to their low power consumption
and adequate processing capabilities.  Projects where an embedded system is the
central processing unit are not common. Due to the fact that traditional
centralized processing systems are used to collect data, there is an excessive
use of IoT devices for transmitting that data.

There are sources that demonstrate that the total amount of user data to be
stored, or processed doubles every two years. The first problems due to this
trend are currently being faced by industry; such is the case of the aircraft
industry.

In 2013, the Virgin Atlantic\textregistered\  airline announced that the Boeing
787\textregistered\  aircraft was going to be one of the first highly connected
planes. Each one of these airplanes was expected to create over half of a
terabyte of data per transatlantic flight. Due to the costs of handling and
transmitting such amount of data daily, the airline started to look for
cloud-based solutions inside the airplanes. This is an expensive solution;
because the cost in space and energy required to handle a cloud-based solution
inside an airplane is too high. 

Current IoT architectures are not efficiently using the computing resources
they have. In addition to this, the lack of standard methodologies to design
distributed self-sustainable networks of IoT systems is causing problems in the
industry. Such and other reasons, which will be discussed in the document,
justify pursuing a quantitative comparison of IoT distributed low power
computing systems to high-performance computing systems in terms of power
efficiency. 

\clearpage
