%Empieza configuracion
\setstretch{1.0}
\titleformat{\chapter}{\Huge\bfseries}{\thechapter}{0 pt}{\rule{340 pt}{3 pt}\\}
\titlespacing{\chapter}{100 pt}{-25 pt}{40 pt}[10 pt]	
\pagestyle{fancy}
\fancyhead[RO,RE]{\thepage}
\fancyfoot[CO,CE]{}
%Termina configuracion

\chapter*{Agradecimientos}
\addcontentsline{toc}{chapter}{Agradecimientos}
\setstretch{1.5} %Regresa el interlineado a 1.5

\normalsize
\noindent En la p'agina de reconocimientos, el autor o la autora expresa sus agradecimientos profesionales y personales; incluyendo los permisos que haya conseguido para utilizar materiales anteriormente publicados y con derechos reservados. Los reconocimientos se redactan de manera profesional, no anecd'otica.

Aseg'urate de ser congruente con la persona en la que redactas los agradecimientos. Si inicias con ``el autor desea...'', tendr'as que usar la tercera persona siempre. Si usas ``deseo, quiero, me gustar'ia agradecer...'', se usar'a la primera persona consistentemente.

En lo que sigue se presenta un ejemplo de agradecimiento:

Quisiera reconocer, antes que nada, el apoyo y paciencia de mi asesor de tesis doctoral, el Dr. Claude Berrou. Su apoyo en la realizaci'on de este formato fue muy valioso. Me benefici'e mucho de pl'aticas que sostuve con 'el durante mis estudios doctorales. Tambi'en quisiera agradecer a los otros miembros del comit'e supervisor de tesis por su tiempo y atenci'on.

Le agradezco al Dr. Guillermo Alfonso Parra Rodr'iguez y al Dr. Jos'e Ram'on Alvarez Bada en la realizaci'on de este documento. Este formato de propuesta de tesis de maestr'ia est'a inspirado de un documento que me envi'o el Dr. Parra, el documento que el Dr. Alvarez utiliz'o previamente, as'i como en el formato de tesis de doctorado que utilic'e para la redacci'on de mi tesis en la Escuela Nacional Superior de Telecomunicaciones, en Brest, Francia.

Quisiera agradecer a mis alumnos del curso Seminario de Innovaci'on y Creatividad su paciencia durante el trimestre conforme iba organizando mis ideas sobre cu'ales son las caracter'isticas de un buen trabajo de investigaci'on as'i como en la mejora de este documento.

Quiero agradecer a los miembros del Departamento de El'ectrica y Electr'onica del ITESM Campus Guadalajara por compartir varios momentos de pl'atica y sana convivencia.
\clearpage