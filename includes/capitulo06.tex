%Empieza configuracion de capitulo
\setstretch{1.0}
\titleformat{\chapter}[block]{\Large\bfseries}{CHAPTER \Huge\thechapter\vspace{25 pt}}{0 pt}{\\\fontsize{26}{36}\selectfont}
\titlespacing{\chapter}{0 pt}{30 pt}{50 pt}[0 pt]
\titleformat{\section}{\Large\bfseries}{\thesection}{0 pt}{\hspace{30 pt}}
\titleformat{\subsection}{\large\bfseries}{\thesubsection}{0 pt}{\hspace{30 pt}}
\pagestyle{fancy}
\fancyhead[LO,LE]{\footnotesize\emph{\leftmark}}
\fancyhead[RO,RE]{\thepage}
\fancyfoot[CO,CE]{}
%Termina configuracion de capitulo

\chapter{Conclusions}
\setstretch{1.5} %Regresa el interlineado a 1.5

\normalsize
\noindent

\section{Conclusions}
\noindent

In this thesis work it was proposed the hypothesis that a cluster of ultra-low
power IoT platforms can be as computing powerful and energy efficient as a
traditional computing system.

In chapter one we present a simulation to illustrate such hypothesis. It is described in
Figure~\ref{fig:1.2}.  At the beginning , the increment in the number of nodes
in the studied network produces an increment on performance; however, it is
expected to reach a maximum point at which power efficiency becomes stable and
it will remain in such state up to a certain point at which it will start to
decrease. 

As we can see in the previous graphs this behavior was achieved as expected. At
the number of platforms increases it is expected the performance benefit
increase, because the amount of work to be done is distributed among different
platforms, but as more are added due to the power they consume the performance
gain starts to minimize. When the ideal number of platforms is exceeded, the
power efficiency decrease rapidly.

In all the experiment we realize, the ideal number of platforms is always
between three and four. This gave us the confidence to say that as a conclusion
that the energy efficiency (using MPI Benchmarks as a reference workload) of an
IoT distributed system is similar to a traditional computing system \cite{NUC}
with three or four nodes.

The side effect that is presented is the latency. The benchmarks that measure
latency present a lower power efficiency as a cluster of embedded platforms
than the traditional desktop system. 

The main objective of this work was to provide to the IoT industry a way to
determine whether applications can take advantage of the communication between
IoT devices to process their own data instead of sending information to data
centers. Based on the results presented we can say that if the IoT system has
at least four or five embedded platform with similar characteristics to the one
presented here \cite{minnowboard} is better to process their own data instead
of sending it to data centers.

\section{Future Work}
\noindent

As we know the IoT revolution lacks of industry standards. There are thousands
of computing devices and sensors from different vendors that appear on the
market every day. There are two main projects that are organizing information
to establish standards for IoT communications (described in Chapter 1).

The AllSeen Alliance (before known as AllJoyn) is one of them . It is an open
source project for the development of a universal software framework aimed at
the interoperability among heterogeneous devices, dynamic creation of proximal
networks and execution of distributed applications. The framework provides a
common interface towards smart devices.

As future work will be worth to try to implement libraries for parallel and
distributed computing as part of AllSeen Alliance standard. With this, the
future IoT products will have the chance to determine whether applications can
take advantage of the communication between IoT devices to process their own
data instead of sending information to data centers. 

\clearpage
