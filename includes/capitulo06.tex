%Empieza configuracion de capitulo
\setstretch{1.0}
\titleformat{\chapter}[block]{\Large\bfseries}{CAP'ITULO \Huge\thechapter\vspace{25 pt}}{0 pt}{\\\fontsize{26}{36}\selectfont}
\titlespacing{\chapter}{0 pt}{30 pt}{50 pt}[0 pt]
\titleformat{\section}{\Large\bfseries}{\thesection}{0 pt}{\hspace{30 pt}}
\titleformat{\subsection}{\large\bfseries}{\thesubsection}{0 pt}{\hspace{30 pt}}
\pagestyle{fancy}
\fancyhead[LO,LE]{\footnotesize\emph{\leftmark}}
\fancyhead[RO,RE]{\thepage}
\fancyfoot[CO,CE]{}
%Termina configuracion de capitulo

\chapter{Experiments}
\setstretch{1.5} %Regresa el interlineado a 1.5

\normalsize
\noindent

\section{Find the correct OS}
\noindent

    The forst aproach was to find the correct Operating system for the test,
    however this wasn't easy in the begining we startd to create a few
    experiments. In the end we came up with a full suport of MPi for embedded
    platfoms suport that the comunity thanks to us a lot.

    \subsection {Embedded Distributed Systems: A Case of Study with Clear Linux
    Project for Intel Architecture}
    \noindent



The main objective of this work will be to prove that a distributed embedded
system (Intel® AtomTM Processor E3825) running real HPC workloads 
(MPI benchmarks) can be improved by the use of a customized operating system 



The need of more complex and smart applications (they must adapt their
performance as well as power) has risen the bar to create distributed systems
based on parallel embedded platforms. 

By definition: A distributed system consists of a collection of autonomous
computers, connected through a network and distribution middle-ware, which
enables computers to coordinate their activities and to share the resources of
the system, so that users perceive the system as a single, integrated computing
facility.

Advantages: 

\begin{enumerate} 
    
    \item \textbf{Partioning Workload}: 
    By partitioning the workload onto multiple processors, 
    each processor is now responsible for only a fraction of the workload. 
    The processors can now afford to slow down by dynamic voltage scaling 
    (DVS) to run at more power-efficient states, and the degraded performance 
    on each processor can still contribute to an increased system-wide 
    performance by the increased parallelism.  

    \item \textbf{Heterogeneous HW}: 
    Another advantage with a distributed scheme is that heterogeneous hardware 
    such as DSP and other accelerators can further improve power efficiency 
    of various stages of the computation through specialisation.

\end{enumerate}
 

Disadvantages: 

\begin{enumerate}
    \item \textbf{Network}: Despite the fact the distributed systems may have
    many attractive properties, they pay a higher price for message-passing 
    communications. Each node now must handle not only communication with 
    the external world, but also extra communication on the internal network. 
    As a result, even if the actual data payload is not large on an absolute 
    scale, the communication appears very expensive and does not scale to a few
    more nodes

    \item \textbf{Lack of optimised OS}: A typical embedded system often does
    not contain an operating system. Crafting distributed programs on such a
    bare-bone platform is extremely difficult and error-prone. Although many
    higher-level abstractions such as Message Passing Interfaces (MPI) have been
    proposed to facilitate distributed programming, these abstraction layers require
    extensive system resources with comprehensive operating systems support, which
    may not be available to an embedded platform 
    
\end{enumerate}
 
However in recent years we have seen an emergence of a new class of full-fledged
embedded systems (they are fully loaded with sufficient system resources as well
as networking and other peripheral devices, and a complete version of the
operating system with network support) In addition, they are typically designed
with power-management technology in order to extend the battery life

With these gaps closed there might be a chance to merge the parallel and
distributed paradigms on the embedded world.  A merging point of technologies
from different domains often inspires technology innovations in new domains.

\section{Development}

According to these in consideration there are multiple scenarios to test the
capability of an embedded distributed system: 


\begin{itemize} 
    
    \item Compare an Embedded system with generic SW (Linux base OS
    (Fedora/Ubuntu/Debian) and generic MPI protocol (MPICH)) against a
    regular development system (with the same OS and MPI toolS)

    \item Compare an Embedded system with a distributed operating system
    against the same embedded system with custom SW (Linux from scratch system)


    \item Compare an Embedded system with a distributed operating system
    against the same embedded system with custom SW (Linux from scratch system and
    MPI for embedded (LMPI)) in order to check the gap in the multiple systems

\end{itemize}

For this report we will execute the experiment of the second scenario, due to the
fact that we have already done the study of the first scenario. In that case we
realize that despite the fact that the minnow Max ran 8 times slower than the
regular development system (NUC Haswell system) the Minnow Max was more stable
and with less drops in performance. 

The operating system we will use is the Fedora 19 system, the description of the
system is listed on the fedora project site home page (http://fedoraproject.org)

The benchmark we will use to measure the performance is MPIbench. This is a
program to measure the performance of some critical MPI functions. By critical
it means that the behavior of these functions can dominate the run time of a
distributed application. MPBench has now been integrated into LLCbench (Low
Level Characterisation Benchmarks) 

The MPIfunctions that it stress are: 


\begin{itemize}

\item MPI\_Send/MPI\_Recv Bandwidth (Kb/second vs. bytes) 
\item MPI\_Send/MPI\_Recv Application latency or Gap time (us vs. bytes)
\item MPI\_Send/MPI\_Recv Roundtrip or 2 * Latency (trns/second vs. bytes) 
\item MPI\_Send/MPI\_Recv() BidirectionalBandwidth (Kb/second vs. bytes) 
\item MPI\_Bcast broadcast (Kb/second vs. bytes) 
\item MPI\_Reduce reduction (sum) (Kb/second vs. bytes) 
\item MPI\_AllReduce reduction (sum) (Kb/second vs. bytes) 
\item MPI\_Alltoall Each process sends to every other process (Kb/sec vs. bytes) 

\end{itemize}


    \subsection {Embedded Distributed Systems: A Case of Study with Yocto project}
    \noindent

\section{Find the correct MPI protocol configuration}
\noindent

\section{Embedded Distributed System: A case of study in smart greenhouses}
\noindent

\section{Embedded Distributed System: PnP measurement with comercial OS}
\noindent

\clearpage
