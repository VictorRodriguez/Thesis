%Empieza configuracion de capitulo
\setstretch{1.0}
\titleformat{\chapter}[block]{\Large\bfseries}{CAP'ITULO \Huge\thechapter\vspace{25 pt}}{0 pt}{\\\fontsize{26}{36}\selectfont}
\titlespacing{\chapter}{0 pt}{30 pt}{50 pt}[0 pt]
\titleformat{\section}{\Large\bfseries}{\thesection}{0 pt}{\hspace{30 pt}}
\titleformat{\subsection}{\large\bfseries}{\thesubsection}{0 pt}{\hspace{30 pt}}
\pagestyle{fancy}
\fancyhead[LO,LE]{\footnotesize\emph{\leftmark}}
\fancyhead[RO,RE]{\thepage}
\fancyfoot[CO,CE]{}
%Termina configuracion de capitulo

\chapter{Conclusions}
\setstretch{1.5} %Regresa el interlineado a 1.5

\normalsize
\noindent

\section{Conclusions}
\noindent

In this thesis work it was proposed the hypothesis that a cluster of ultra-low
power IoT platforms can be as computing powerful and energy efficient as a
traditional computing system.

In chapter one we present a simulation to illustrate such hypothesis. It is described in
Figure~\ref{fig:1.2}.  At the beginning , the increment in the number of nodes
in the studied network produces an increment on performance; however, it is
expected to reach a maximum point at which power efficiency becomes stable and
it will remain in such state up to a certain point at which it will start to
decrease. 

As we can see in the previous graphs this behavior was achieved as expected. At
the number of platforms increases it is expected the performance benefit
increase, because the amount of work to be done is distributed among different
platforms, but as more are added due to the power they consume the performance
gain starts to minimize. When the ideal number of platforms is exceeded, the
power efficiency decrease rapidly.

In all the experiment we realize, the ideal number of platforms is always
between three and four. This gave us the confidence to say that as a conclusion
that the energy efficiency (using MPI Benchmarks as a reference workload) of an
IoT distributed system is similar to a traditional computing system \cite{NUC}
with three or four nodes.

\section{Future Work}
\noindent

\clearpage
