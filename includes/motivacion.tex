%Empieza configuracion
\setstretch{1.0}
\titleformat{\chapter}{\Huge\bfseries}{\thechapter}{0 pt}{\rule{340 pt}{3 pt}\\}
\titlespacing{\chapter}{100 pt}{-25 pt}{40 pt}[10 pt]	
\pagestyle{fancy}
\fancyhead[RO,RE]{\thepage}
\fancyfoot[CO,CE]{}
%Termina configuracion

\chapter*{Motivation}
\addcontentsline{toc}{chapter}{Motivation}
\setstretch{1.5} %Regresa el interlineado a 1.5

\normalsize
\noindent
Every day we are in contact with computing and sensing devices everywhere. Our
daily activities, for example exercising, could be tracked by multiple sensors
that measure heart rate, distance, speed and consumed calories. These devices,
which are provided with unique identifiers and the ability to transfer data
over a network, are part of the Internet of Things (IoT) technology.

IoT systems are more than just computing and sensing devices. They include
storage and information processing for sensors to display meaningful
information on an easy to access user interface. 

The use of IoT devices is rising quickly, as well as its problems. According to
\cite{Benkhelifa} by 2022 it is expected to have 14 billions of IoT devices
running in the world. The rise of IoT will lead to an explosion in the volume
of data transmitted, collected and processed. In addition, the power
consumption of computing systems that store and process data generated by IoT
devices is a very important design restriction to make these systems
successful. If current trends continue, a system to handle IoT data would
require 100 megawatts \cite{Xizhou}.

However, the computational power that the IoT systems have might be enough to
process all the data they generate. As theoretical physicist Michio Kaku
mention in \cite{Michio}: \textit{"Today, your cell phone has more computer
power than all of NASA back in 1969, when it placed two astronauts on the
moon"}. This fact motivated this work to investigate whether if it is possible
that a network of IoT systems could process the data they generate on their own
at real time. In order to answer that question it is necessary to make a
technical comparison between a network of IoT systems and a traditional
computing system used for storage and processing of data coming from IoT
devices. 

This thesis focuses on the development of a methodology to determine if it is
necessary to send sensors data to a centralized processing system, or if it can
be managed by a network of IoT computing systems. 

\clearpage
