%Empieza configuracion
\setstretch{1.0}
\titleformat{\chapter}{\Huge\bfseries}{\thechapter}{0 pt}{\rule{340 pt}{3 pt}\\}
\titlespacing{\chapter}{100 pt}{-25 pt}{40 pt}[10 pt]	
\pagestyle{fancy}
\fancyhead[RO,RE]{\thepage}
\fancyfoot[CO,CE]{}
%Termina configuracion

\chapter*{Motivation}
\addcontentsline{toc}{chapter}{Motivation}
\setstretch{1.5} %Regresa el interlineado a 1.5

\normalsize
\noindent

We live in a world full of compute and sensing  devices, not just the personal
computers we use for work or school. Everyday we are in contact with compute
and sensing devices everywhere. Just in our homes the automatic doors, smart
televisions and fire alarms helps us in our daily activities. The early morning
jog is tracked by sport watches that measure our distance, speed and calories
consumed during the exercise. All these devices provided with unique identifiers 
and the ability to transfer data over a network are part of the Internet of
Things(IoT) technology. 

The IoT systems are much more than just the compute and sensing devices. There
must be a part that storage and process the information gathered from all the
sensors  in order to display the meaningfull information on an easy to access
user interface. 

The use of IoT devices is rising quickly. According to the number of connected
devices by 2022 is estimated to 14 billion\cite{Benkhelifa}. As you can imagine
the rise of IoT will lead to an explosion in the volume of data collected,
transmitted and processed.

Besides that, the power consumption of the compute systems that storage and
process the data generated by the IoT devices is a key part to considerate. If
current trends continue, a system to storage and process the IoT (a petaflop
system) data will require 100 megawatts. 



This motivate us to wonder if is possible to create an IoT system that does
not need a server. After an initial research we found that there isn't a
technical comparison of an  IoT system and a traditional computing system
using as server. 

Answering these questions will help others in the embedded insdustry to apply
the same aproach for their applications. The final goal will be to generate a
starndard methodology in the IoT industry that helps the new projects to
determine if they need a centralized  server or they can handle their data
processing by themselsf , minimizing the volume of data transmited and
processed on external servers. 


\clearpage
