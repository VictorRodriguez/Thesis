%Empieza configuracion
\setstretch{1.0}
\titleformat{\chapter}{\Huge\bfseries}{\thechapter}{0 pt}{\rule{340 pt}{3 pt}\\}
\titlespacing{\chapter}{100 pt}{-25 pt}{40 pt}[10 pt]	
\pagestyle{fancy}
\fancyhead[RO,RE]{\thepage}
\fancyfoot[CO,CE]{}
%Termina configuracion

\chapter*{Motivation}
\addcontentsline{toc}{chapter}{Motivation}
\setstretch{1.5} %Regresa el interlineado a 1.5

\normalsize
\noindent


What mas the main motivation to start this work? Lets start with facing a fact
Computers are everywhere , this fact, of course, is not a surprise to anyone
who hasn't been living in a cave during the past 25 years or so. And you
probably know that computers aren't just on our desktops, in our kitchens, and,
increasingly, in our living rooms. Few years ago there was a change in this
fact: Computers are everywhere and they are interconnected to each other.  This
new paradigm is known as Internet of Things(IoT). 

Nowadays we are seeing environment in which multiple objects are provided with
unique identifiers and the ability to transfer data over a network. But the IoT
systems are much more than just the sensors. There must be a part that storage
and process the information gathered from all the sensors (server) in order to
display the meaning full information on an easy to access user interface. 

As you can imagine the rise of IoT will lead to an explosion in the volume of
data collected, transmitted and processed. Besides that, the power consumption
of  is a key part to considerate. If current trends continue, a system to
storage and process the IoT (a petaflop system) data will require 100
megawatts. 

This motivate us to wonder if is possible to create a IoT system that does
not need a server. After an initial research we found that there isn't a
technical comparison of an  IoT system and a traditional computing system
using as server. 

Answering this questions will help others in the embedded insdustry to apply
the same aproach for their applications. The final goal will be to generate
an starndard metodology in the IoT industry that helps the new projects to 
determine if they need a centirlized server or they can handle their data processing by
themselsfs , minimizing the volume of data transmited and processed on external
servers. 


\clearpage
