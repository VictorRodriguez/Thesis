%Empieza configuracion
\setstretch{1.0}
\titleformat{\chapter}{\Huge\bfseries}{\thechapter}{0 pt}{\rule{340 pt}{3 pt}\\}
\titlespacing{\chapter}{100 pt}{-25 pt}{40 pt}[10 pt]	
\pagestyle{fancy}
\fancyhead[RO,RE]{\thepage}
\fancyfoot[CO,CE]{}
%Termina configuracion

\chapter*{Motivation}
\addcontentsline{toc}{chapter}{Motivation}
\setstretch{1.5} %Regresa el interlineado a 1.5

\normalsize
\noindent

We live in a world full of compute and sensing  devices, not just the personal
computers used for work or school. Every day we are in contact with compute
and sensing devices everywhere. Just in the early morning jogging our activity
cloud be tracked by multiple sensors that measure our distance, speed and calories
consumed during the exercise. All these devices provided with unique identifiers 
and the ability to transfer data over a network are part of the Internet of
Things(IoT) technology. 

The IoT systems are much more than just the compute and sensing devices. There
must be a part that storage and process the information gathered from all the
sensors  in order to display the meaningful information on an easy to access
user interface. 

The use of IoT devices is rising quickly, as well as its problems. According to
\cite{Benkhelifa} by 2022 is expected to have 14 billions of IoT devices. As
you can imagine the rise of IoT will lead to an explosion in the volume of data
collected, transmitted and processed. Besides that, the power consumption of
the compute systems that storage and process the data generated by the IoT
devices is a key part to considerate. If current trends continue, a system to
handle the IoT data will require 100 megawatts. \cite{Xizhou}

However the computational power that the IoT systems might be enough to
process all the data they generate. As  theoretical physicist Michio Kaku
mention in \cite{Michio}: \textit{"Today, your cell phone has more computer
power than all of NASA back in 1969, when it placed two astronauts on the
moon"}.

This fact motivate us to wonder if is possible that a network of IoT systems
could process the data they generate by their own. In order to answer that
question is necessary to make a technical comparison between a network of  IoT
systems and a traditional computing system used for storage and processing of
data incoming from IoT devices. 

The main objective of this work is to provide a methodology that helps others
in the IoT industry to determine if is necessary to send the sensor's data to a
centralized processing system or it can be managed by a network of IoT compute
systems. 

\clearpage
