%Empieza configuracion de capitulo
\setstretch{1.0}
\titleformat{\chapter}[block]{\Large\bfseries}{CAP'ITULO \Huge\thechapter\vspace{25 pt}}{0 pt}{\\\fontsize{26}{36}\selectfont}
\titlespacing{\chapter}{0 pt}{30 pt}{50 pt}[0 pt]
\titleformat{\section}{\Large\bfseries}{\thesection}{0 pt}{\hspace{30 pt}}
\titleformat{\subsection}{\large\bfseries}{\thesubsection}{0 pt}{\hspace{30 pt}}
\pagestyle{fancy}
\fancyhead[LO,LE]{\footnotesize\textit{\leftmark}}
\fancyhead[RO,RE]{\thepage}
\fancyfoot[CO,CE]{}
%Termina configuracion de capitulo

\chapter{Marco Te'orico} %Cambia Marco Te'orico al nombre de tu capitulo
\setstretch{1.5} %Regresa el interlineado a 1.5

\normalsize

\section{Contenido del marco te'orico}
\noindent
Normalmente, el cap'itulo 2 del trabajo de investigaci'on lleva el marco te'orico, es decir, aqu'ellas cosas que ya se saben sobre el tema a investigar, que son de dominio p'ublico y que por tanto se deben de mencionar antes de presentar propiamente la propuesta de investigaci'on (o los resultados del proyecto, cuando se trate de la tesis definitiva).

\subsection{Caracter'isticas generales del marco te'orico}
\noindent
Es muy probable que en alg'un momento usted deba presentar alguna ecuaci'on, tal vez generada por usted o encontrada en alguna fuente. Cuando usted incluya ecuaciones en su documento, identif'iquelas con un par de n'umeros entre par'entesis, puestos despu'es de la ecuaci'on, de tal forma que el primer n'umero represente el cap'itulo y el segundo n'umero identifique el orden de la ecuaci'on en el cap'itulo. Referirse a la secci'on 4.3 para ver m'as detalles.

En el marco te'orico, se recomienda seguir la norma: ``poco, pero selecto'' \cite{Demo:comunicacion}. Esto quiere decir que uno debe dar toda la informaci'on relevante necesaria para entender el tema de investigaci'on, sin saturar al lector de informaci'on trivial y poco relacionada con el tema pero sin omitir puntos importantes. Se debe ser claro y conciso en la expresi'on.

\subsection{Algunas recomendaciones sobre el marco te'orico}
\noindent
En algunos casos, en los que una secci'on de un cap'itulo se necesite subdividir en temas a'un m'as espec'ificos, se pueden usar subsecciones como 'esta, en la que el t'itulo (que comienza con un n'umero como 2.1.2) se escribe en negritas con letra Times de tama'no 12.

Para el marco te'orico, como para los dem'as cap'itulos, uno debe escribir con sus propias palabras el texto, reservando las palabras de otros autores s'olo para citas expl'icitas, que deben siempre llevar la correspondiente referencia bibliogr'afica \cite{Demo:tesis_autor}.

Entre el final de una secci'on y la secci'on siguiente se debe dejar una l'inea en blanco, tal y como ocurre a continuaci'on.

\section{Formato del marco te'orico y dem'as cap'itulos}
\noindent
A continuaci'on se presenta informaci'on general sobre el formato de los diferentes cap'itulos de la propuesta o del documento final. Para mayor detalle del formato referirse al cap'itulo 4. El texto debe escribirse en texto totalmente justificado con letra Times New Roman de tama'no 12. Los t'itulos de los cap'itulos se ponen en la esquina superior izquierda de la p'agina en que comienzan y deben usar tama'no de letra 24. Los t'itulos de secci'on (2.1, 2.2, etc.) deben usar tama'no de letra 18, siempre en Times New Roman.

Con excepci'on del encabezado de referencias, todos los t'itulos deben ir a la izquierda de la p'agina. El texto debe estar justificado y centrado en la p'agina, dejando m'argenes de 3 cms a la izquierda y 2 cms a la derecha de la p'agina y 2 cms arriba y debajo de cada p'agina sin considerar el encabezado y pie de p'agina. No se olvide de numerar las p'aginas en la parte inferior central de cada inicio de cap'itulo y en la parte superior derecha para las siguientes p'aginas.


\clearpage