%Empieza configuracion de capitulo
\setstretch{1.0}
\titleformat{\chapter}[block]{\Large\bfseries}{CHAPTER \Huge\thechapter\vspace{25 pt}}{0 pt}{\\\fontsize{26}{36}\selectfont}
\titlespacing{\chapter}{0 pt}{30 pt}{50 pt}[0 pt]
\titleformat{\section}{\Large\bfseries}{\thesection}{0 pt}{\hspace{30 pt}}
\titleformat{\subsection}{\large\bfseries}{\thesubsection}{0 pt}{\hspace{30 pt}}
\pagestyle{fancy}
\fancyhead[LO,LE]{\footnotesize\textit{\leftmark}}
\fancyhead[RO,RE]{\thepage}
\fancyfoot[CO,CE]{}
%Termina configuracion de capitulo

\chapter{Objectives} %Cambia Marco Te'orico al nombre de tu capitulo
\setstretch{1.5} %Regresa el interlineado a 1.5

\normalsize

\section{General Objective}
\noindent

The main objective is to present a methodlogy to crate a self sustainable
network of ultra-low-voltage microprocessors platforms. It means the embedded 
devices interconected could process their own workload without the need of an 
external compute system

The purpose of this methodology is to give an experienced IoT  developer 
enough information to replicate the study. At the same time it offers the theoretical 
underpinning for understanding which method, set of methods, or so-called “best 
practices” can be applied an specific case


\section{Justification}
\noindent

The development of wireless sensor networks has reached a point where each 
individual node of a network may store and deliver a massive amount of 
(sensor-based) information at once or over time. Right now the total amount of 
user data (data payload) to be stored or processed doubles every two years. Consequently,
data will become a problem for traditional data aggregation strategies 
traffic-wise as well as with regard to energy efficiency. 

These problems start to be relevant in current industries. Such is the case of 
the Aircraft industry. On March of 2013 Virgin Atlantic was preparing for a 
significant increase in data as it embraces the internet of things, with a new fleet of highly 
connected planes each expected to create over half a terabyte of data per flight.

Speaking to the Computerworld UK magazine at the Economist Technology Frontiers 2103 event, Virgin 
Atlantic IT director David Bulman said that the airline company was expecting an 
"explosion" of information generated from a growing number of sources, from 
employees and customers to cargo containers and planes.

In particular, the introduction of Boeing 787 aircraft ordered by Virgin 
Atlantic for delivery in 2014  was expected to dramatically increase the volume 
of data the airline will need to deal with.

From the interview Bulman hightlight their current problems: 

\say{The challenge is what do you do with that amount of data when you are getting 
terabytes of data a day off your various airplanes? We are getting to the stage 
right now where we cannot deal with that much}

He added: 

\say{If you are talking that level of data you can't just chuck ten disks 
into your data centre anymore, you have to look at cloud based solutions and how 
you can store data.}

As we can see the lack of standard solutions for self sustainable networks is 
creating real problems among the industry


\clearpage
