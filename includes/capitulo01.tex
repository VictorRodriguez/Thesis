%Empieza configuracion de capitulo
\setstretch{1.0}
\titleformat{\chapter}[block]{\Large\bfseries}{CAP'ITULO \Huge\thechapter\vspace{25 pt}}{0 pt}{\\\fontsize{26}{36}\selectfont}
\titlespacing{\chapter}{0 pt}{30 pt}{50 pt}[0 pt]
\titleformat{\section}{\Large\bfseries}{\thesection}{0 pt}{\hspace{30 pt}}
\titleformat{\subsection}{\large\bfseries}{\thesubsection}{0 pt}{\hspace{30 pt}}
\pagestyle{fancy}
\fancyhead[LO,LE]{\footnotesize\textit{\leftmark}}
\fancyhead[RO,RE]{\thepage}
\fancyfoot[CO,CE]{}
%Termina configuracion de capitulo

\chapter{Introducci'on} %Cambia Introducci'on al nombre de tu capitulo
\setstretch{1.5} %Regresa el interlineado a 1.5

\normalsize

\section{Antecedentes}
\vspace{30 pt}
\noindent
Imaginamos que, en estos momentos, usted ya lleva muchas horas estudiando y consultando documentos sobre su tema de tesis de maestr'ia. Ya defini'o un 'area de trabajo y ha le'ido cuando menos diez o quince art'iculos t'ecnicos relacionados con su tema. En este manual se le va a dar informaci'on importante sobre la forma en que usted debe redactar la propuesta de tesis que representar'a el rubro m'as importante en su calificaci'on del Seminario de Innovaci'on y Creatividad.\\

Este manual ha sido dise'nado siguiendo un formato similar al que deber'a usted presentar en su propuesta de tesis. La tesis de maestr'ia ser'a redactada, con algunas variaciones, de la misma forma que la propuesta de tesis, por lo que la realizaci'on de este documento representar'a una buena pr'actica. En esta secci'on se redactan los antecedentes de su proyecto. Los antecedentes son el primer contacto con el lector, por lo que es importante explicar los motivos que despertaron su inter'es en la investigaci'on. Describa la necesidad que lo llev'o a realizar la investigaci'on sobre el tema, c'omo se interes'o en ella, d'onde, cu'ando y qu'e o qui'en lo estimul'o para que lo llevara a cabo \cite{Demo:tesis_autor}.

\section{Definici'on del problema}
\noindent
As'i como la primera secci'on del primer cap'itulo trata los antecedentes del proyecto de investigaci'on, la segunda secci'on de este cap'itulo proporciona la definici'on del problema. Despu'es de haber le'ido suficiente bibliograf'ia sobre el tema, ser'a m'as f'acil definir el problema de su investigaci'on.

Lo primero que se debe lograr es definir correctamente el problema pues exponerlo de manera vaga origina cuestionamientos irrelevantes que nos desv'ian de los objetivos de la investigaci'on \cite{Demo:tesis_autor}. Considere que su problema surge de una idea, una dificultad, una necesidad o una duda. En esta secci'on es importante establecer la delimitaci'on o alcance del mismo ya sea en funci'on de tiempo, espacio o recursos.

\section{Objetivos}
\noindent
Sea claro y cuidadoso en la redacci'on de su objetivo u objetivos. Recuerde que lo que escriba aqu'i debe ser razonablemente ambicioso pues es una tesis de posgrado, pero al mismo tiempo debe ser realizable. Se recomienda comenzar con un peque'no pre'ambulo explicando qu'e es importante del problema, antes de proceder a describir, en lenguaje sencillo, cu'ales son los objetivos del proyecto. Los objetivos deben tratar sobre ``qu'e es lo que usted va a ofrecer'' durante su investigaci'on, es decir, los resultados que pretende obtener durante su trabajo.

En la redacci'on de su documento, procure ser lo m'as consistente posible, usando siempre la misma notaci'on para los mismos conceptos. Se puede escribir en negritas o it'alicas en el texto, tablas y figuras pero procure usar siempre el mismo tipo de letra para los mismos tipos de conceptos. Los pies de figura van, como es de esperar, debajo de la figura correspondiente. Utilice una numeraci'on doble punteada para identificar a cada dibujo, con el primer n'umero identificando al cap'itulo y el segundo n'umero identificando al dibujo particular en ese cap'itulo (es decir, 1.4, 1.5, etc.).

\section{Justificaci'on}
\noindent
En esta secci'on explique por qu'e piensa usted que el proyecto es importante. Incluya argu- mentos que convenzan al lector de la relevancia del tema y del proyecto. En algunos casos, puede usar argumentos de tipo intelectual. En otros, tal vez sea m'as convincente apelar a consideraciones econ'omicas o de 'indole m'as pr'actica.

\section{Hip'otesis}
\noindent
En la secci'on de hip'otesis debe especificar cu'ales son las cosas que se est'an asumiendo como punto de partida en el proyecto de investigaci'on. Para que el proyecto de investigaci'on tenga valor, deben haber supuestos que no se saben de fijo y que se busca verificar o falsificar (por ejemplo, se busca saber si Bluetooth es una tecnolog'ia con m'as o menos mercado potencial que WiFi).

La hip'otesis debe por lo tanto ser un enunciado o grupo de enunciados que puedan ser demostrados o refutados de manera razonablemente confiable. Es importante que establezca c'omo se van a probar o refutar las hip'otesis planteadas, lo cual se hace en las secciones de planeaci'on.

\section{Metodolog'ia}
\noindent
En esta secci'on debe hacer un bosquejo de la manera en que se propone llevar a cabo la investigaci'on. Cuanto m'as completo el bosquejo, m'as f'acil se desarrollar'a el proceso de investigaci'on. En esta secci'on tiene que explicar lo que se va a realizar para lograr el objetivo de la investigaci'on, c'omo lo har'a, con qu'e elementos cuenta, qu'e equipos o programas se necesitan \cite{Demo:tesis_autor}.

\clearpage