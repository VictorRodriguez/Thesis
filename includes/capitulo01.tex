%Empieza configuracion de capitulo

\setstretch{1.0} \titleformat{\chapter}[block]{\Large\bfseries}{CHAPTER
\Huge\thechapter\vspace{25 pt}}{0 pt}{\\\fontsize{26}{36}\selectfont}
\titlespacing{\chapter}{0 pt}{30 pt}{50 pt}[0 pt]
\titleformat{\section}{\Large\bfseries}{\thesection}{0 pt}{\hspace{30 pt}}
\titleformat{\subsection}{\large\bfseries}{\thesubsection}{0 pt}{\hspace{30
pt}} \pagestyle{fancy} \fancyhead[LO,LE]{\footnotesize\textit{\leftmark}}
\fancyhead[RO,RE]{\thepage} \fancyfoot[CO,CE]{}
%Termina configuracion de capitulo

\chapter{Introduction} %Cambia Introducci'on al nombre de tu capitulo
\setstretch{1.5} %Regresa el interlineado a 1.5

\normalsize This work presents a detailed study of how within a network of
embedded systems, computing devices collaborate within each other to solve
parallel problems. At the end a detailed case of study based on industrial
benchmarks for distributed systems, power consumption analysis and optimal
operating system selection, describes a methodology to maximize power
efficiency.

\section{Background} \vspace{30 pt} \noindent

The evolution of computing technology has made an incredible progress since the
first general-purpose electronic computer was created. Today, less than 300
dollars will purchase a personal computer that has more performance, more
memory, and more disk storage than a computer bought in 1985 for 1 million
dollars\cite{Hennessy}. Incredible advances have been possible thanks to
innovations in computer and software design. 


%Computter architecture
Since the commercial use of computers (started in 1951 with the introduction of
UNIVAC \cite{Nur}) the development of computers has had multiple changes, not
only from the architectural point of view but also from the application point
of view. In \cite{Hennessy} the authors present the evolution of the computers
in term of performance over the years since 1980. The first important
breakpoint in the history of computers came in the beginning 1970s whit the
emergence of the microprocessor. 

The 1980s saw the rise of the desktop computer based on microprocessors, in the
form of both personal computers and workstations. The 1990s saw the emergence
of the Internet and the World Wide Web, the first successful laptop computing
devices, and the emergence of high performance computing systems for business
purposes (servers). During these years the performance of the microprocessors
increase around 27\% each year \cite{Hennessy}.

However, in the first years of the 2000 decade a dramatic change in the
computer architecture happened. The idea of increasing computing performance in
processors, based mainly on increasing the clock frequency, could not longer be
held. This is due to the fact that the power density (amount of power per
volume unit) is multiplied with increases in frequency. 

The solution, to keep meet Moore's \cite{Mack} law, was shifting to real parallelism by
doubling the number of processors on the die. This was the birth of multi-core
microprocessors area. The idea considered, increasing the computing performance
by limiting the power density at the same time. This new computer architecture
broke the paradigm of using small microprocessors for embedded platforms.
Instead of this it was possible to have more computing power with less
frequency. Thanks to this radical change there has been a rapid evolution of
the computing and multimedia capabilities on embedded systems. 


%Embedded architecture
According to \cite{Hallinan}  \textit{"An embedded system is a special-purpose
system in which the computer is completely encapsulated by the device it
controls"} Unlike a general-purpose computer, an embedded system performs
pre-defined tasks, usually with very specific requirements. Examples of these
are: microwave ovens, washing machines, printers, and GPS (Global Positioning
System) systems. All those electronic based appliances that started to emerge
around 35 years ago \cite{Nur}. 

The variety of embedded applications requires a wide spread of processing power
and cost. They include 8-bit and 16-bit processors that may cost few cents,
32-bit microprocessors that execute 100 million instructions per second and
cost less than few dollars, and high-end processors for the newest video games
or network switches that cost at least 100 dollars and can execute one billion
of instructions per second \cite{Hennessy}.

%Ubicuos systems
Continuous design tradeoffs of factors like: cost, size, power density,
performance and connectivity has caused the computing technology to evolve
rapidly, today ubiquitous computing is a reality. According to Mark
\cite{Mark}, ubiquitous computing is \textit{"the method of enhancing computer
use by making many computers available throughout the physical environment, but
making them effectively invisible to the user"}. This means that computing
power could be made available anywhere and at any time. 


%IoT
According to \cite{Nur}, there is a transformation from ubiquitous computing to
advanced ubiquitous computing. Advanced ubiquitous computing is an extension of
ubiquitous environment that improves connectivity between devices. The main
characteristics of such environment can be listed as follows: a large number of
heterogeneous devices, new communication technology, and Internet of Things
(IoT) among others.


One of the most accurate definitions of IoT given by \cite{Bahga} where it
mentions that \textit{"Internet of Things refers to physical and virtual
objects that have unique identities and are connected to the Internet to
facilitate intelligent applications"}. IoT enables interconnection, via the
Internet, of computing devices embedded in everyday usage objects, enabling
them to send and receive data. Differences respect to traditional embedded
systems include Internet connectivity and smaller power consumption. IoT
systems must always be connected to the Internet and require lower power
consumption.
 An IoT solution has the following parts (figure~\ref{fig:1.1}).


\begin{itemize} 

\item The Thing (computing devices): in the Internet of Things, a thing can be
any natural or man-made object that can be assigned a unique identifier and
provided with the ability to transfer data over a network. 

\item Network Connection: Network Connections provide connectivity between
computing devices and the Internet, a network, or another computing device. 

\item Cloud Computing data centers for storage and big data analysis: The data
by itself is not useful to the end user. After data is sent and stored into the
cloud computing data centers is necessary to run big data solutions that
present meaningful information to users. 

\item Presentation Devices: Dashboards have to be hosted on some kind of
display, it could be a desktop computer running an application, a tablet, or a
smart phone accessing to a web page. It could even be a purpose-built device
like a retail kiosk, an intelligent vending machine or a control panel. The
goal is to present information coming from the big data analysis. 

\end{itemize}

\begin{figure}[H] \centering
\includegraphics[width=1\textwidth]{images/IoT_diagram.jpg} \caption{An IoT
system diagram } \label{fig:1.1} \end{figure}

The IoT revolution lacks of industry standards. There are thousands of
computing devices and sensors from different vendors that appear on the market
every day, each one with its unique way to send or store data in centralized
cloud computing centers. 

There are two main projects that are organizing information to establish
standards for IoT communications: 

\begin{itemize}

\item Open Connectivity Foundation (OCF)\cite{Terry}: The OCF tries to create a
set of open specifications and protocols to enable devices from a variety of
manufacturers to securely and seamlessly interact with one another. Regardless
of the manufacturer, operating system or chipset the devices that adhere to the
OCF specifications should communicate together. 

\item AllSeen Alliance (before known as AllJoyn) \cite{Massimo}: Is an open
source project for the development of a universal software framework aimed at
the interoperability among heterogeneous devices, dynamic creation of proximal
networks and execution of distributed applications. The framework provides a
common interface towards smart devices.

\end{itemize}

Both standards try to solve a simple problem: To establish communications
standards among the industry. This means that all IoT devices could communicate
with each other in spite of the vendor type. Regardless of these, none has
defined a common standard in terms of IoT technology. 

On the other side we daily use computerized technology like: air conditioners,
televisions and cars; many of them connected to the Internet, In spite of the
existing efforts to develop standard network protocols for IoT systems there is
no one to make the IoT systems analyze their data on their own , instead of
sending all data to cloud data centers for analysis. This is a problem in the
short term because the solution might require to add another server (which may
have space and economic constrains).



\section{Problem Definition} \noindent

The fact that use of IoT devices is rising quickly, ( according to
\cite{Benkhelifa} by 2022 is expected to have 14 billions of IoT devices ) is
creating a rise of data processed, transmitted and stored. Taking the scenario
where all these IoT devices try to send 1 kilobyte of data to the centralized
servers at the same time; this will create so much traffic that might be
similar to a security attack \footnote{ In the DDoS attack, high amount of
network traffic with maximum performance are generated and transmitted  to the
target systems\cite{Yang})}, collapsing the centralized data centers.

Apart from the problems of data transmitted, the power consumption of the
compute systems that storage and process the data generated by the IoT devices
is a key part to considerate. If current trends continue, a system to handle
the IoT data will require 100 megawatts. \cite{Xizhou}. 

Despite the variety of IoT applications, \cite{Liu-Dan} \cite{Du}, all of them
are based on the same IoT principle of send data over the internet to a
centralized processing system without using the processing capabilities of the
IoT device.  In many IoT designs \cite{Du} is used a dedicated microprocessor
just to transmit data to the centralized control system; meanwhile, the
intelligence in this centralized control system is just a simple conditional
rule. On the other hand there are very few examples \cite{Wun} where the
compute power of the embedded platforms is exploited. If the IoT industry
continue creating solutions without the appropriate use of the compute power of
the embedded platform and sending all the data over the Internet; soon there
will be more critical examples as the Boeing 787 \footnote{ The Boeing 787
aircraft ordered by Virgin Atlantic for delivery dramatically increases the
volume of data the airline will need to deal with (half terabyte in a
transatlantic flight). Because they can't handle that much terabytes of data
everyday coming from various airplanes they are looking for adding servers
inside the airplanes \cite{Virgin}}

This research work tries to solve the problem of the correct use (one that take
advantage of the compute power) of all the IoT platforms in order to reduce the
amount of data transmitted as well as the amount of high performance servers
needed for data processing. 


\section{Main Objective} \noindent

The main objective of this work is to show how an IoT networks could be self
sustainable. It means to make them solve their own compute problems without the
need to send millions of data to the cloud data centers. The work will provide
a way know when is really necessary to send the data to the cloud data centers.
It will be based on the maximum number of embedded platforms that provides the
maximum level of performance with the less amount of power consumption. At the
same time a list of applications that are good candidates for this approach is
presented. 

This work will provide the IoT industry a way  to determine if their
applications can take advantage of the communication betwen their IoT devices
to process their own data instead of sending the information to data centers.

\section{Hypothesis} \noindent

As we have seen unsustainable power consumption have driven the microprocessor
industry to integrate multiple cores on a single die, or multicore, as an
architectural solution in order to increase the performance. The same approach
might be followed in the current IoT problem recently described. Creating a
self sustainable network of IoT systems, where the processing problems that
they have could be solved among each others with parallel computing technology
(communication among distributed clusters)

With the current compute power of the ultra-low-voltage microprocessors
platforms (core systems of the IoT devices) is possible to create a cluster
with the optimal number of embedded platforms. All of the inter-connected in a
network that provides the maximum level of performance with the less amount of
power consumption. This characteristic is determined by the power efficiency of
the network. The power efficiency is quantified by performance per watt
\cite{Jun}

The critical part is to determine the breaking point where is better to send
the data to the cloud data centers. How many systems is the maximum that these
kind of network could support and still being a good option in terms of energy
efficiency

The development of metrics to evaluate energy efficiency on the basis of
performance and power models is described in \cite{Dong}. According to
\cite{Dong} the formula for performance per watt (Perf/W), which represents the
performance achievable at the same cooling capacity, based on the average power
is as described in \ref{eq:1}:

\begin{equation}\label{eq:1} \frac{Perf}{W} = \frac{1}{(1 + (n -1 ) k (1 - f))}
\end{equation}

Where \textit{n} is the number of processors,  \textit{f} is the fraction of
computation that programmers can parallelize  ( form 0 to 1 ) and \textit{k},
to represent the fraction of power the processor consumes in idle state  ( from
0 to 1 )

In \cite{Dong} their analysis clearly demonstrates that a symmetric many-core
processor can easily  lose its energy efficiency as the number of cores
increases. To achieve the  best possible energy efficiency, their  work
suggests a many-core alternative, featuring many small, energy-efficient cores
integrated with a full-blown processor. They also show that by knowing the
amount of parallelism available in an application prior to execution, is
possible to  find the optimal number of active cores for maximizing performance
for a given cooling capacity and energy in a system

Is because of this that we have the hypothesis that the energy efficiency in a
cluster of ultra-ultra low power platforms will have a similar behavior that
the one presented in \cite{Dong} , with the difference that \textit{n} will be
the number of ultra-low power platforms instead of cores. 

A simulation to illustrate the hypothesis is described in figure~\ref{fig:1.2}.
In the beginning the increment of the number of nodes in our network will
increment the performance (the top part of the equation), but at the same time
the amount of watts will increase making the energy efficiency flat at some
point (if the lower part of the equation increases the equation tends to
decrease)

\begin{figure}[H] \centering
\includegraphics[width=0.75\textwidth]{images/hypothesys.png}
\caption{Hypothesis of energy efficiency behavior in embedded cluster}
\label{fig:1.2} \end{figure}

After finding these curves for command benchmarks it will be easy for the
industry of IoT systems to determine if their applications can take advantage
of communicate their IoT devices among each others instead of sending the
information to their data centers.


\section{Methodology} \noindent

Recent researches \cite{Saldana} \cite{Abgaria} \cite{McMahon} \cite{Liu} are
showing an increasing interest in the topic.  All these research as always talk
about the lack of three parts: 

\begin{itemize} \item A low-voltage microprocessors platform with enough
compute power.  \item An operating system for Distributed Systems.  \item A
light communication protocol to distribute the workload among the embedded
platforms.  \end{itemize}

The way we are going to address this will be:


\begin{itemize}

\item Choose the right embedded platform: There are dozens of embedded and IoT
platforms, this is why is necessary to make a deep analysis and choose the best
platform that feed our needs. (taking into consideration that sometimes the
systems might have heterogeneous platforms)

\item Choose the right communication and compute protocol: There are different
kinds of distributed compute protocols. Part of this investigation is to detect
the most reliable and suitable for our needs.

\item Choose the right performance benchmark. There are different kind of
benchmark in the cluster technology. Is necessary to find the benchmark that
cover the majority of the possible IoT applications.

\item Choose the right Operating System for the system. Once we have selected
the appropriate embedded platforms, another variable in this investigation is
the number of Operating Systems. Either if it is a micro kernel or a monolithic
kernel architecture there are more than a dozen of solutions to use.

\item Create embedded clusters to measure energy efficiency. Once we have found
the best configuration (Hardware + Operating System + Communication Protocol )
in terms of energy efficiency , we can start to create a cluster of embedded
systems.

\item Find the optimal number of embedded systems in the cluster that provide
the highest power efficiency. 

\item Release all the improvements and disagreements found as Open Source. All
the improvements made into any technology (operating systems or communication
protocols) will be published with an open source license.

\item Implement solution on real application ( greenhouse ). In order to test
the hypothesis in a real application we will implement it on a real greenhouse.
Proving that the solution give an embedded system the power of reliability and
availability without the need of external and expensive servers.  \end{itemize}

\clearpage
