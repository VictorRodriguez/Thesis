%Empieza configuracion
\setstretch{1.0}
\titleformat{\chapter}{\Huge\bfseries}{\thechapter}{0 pt}{\rule{340 pt}{3 pt}\\}
\titlespacing{\chapter}{100 pt}{-25 pt}{40 pt}[10 pt]	
\pagestyle{fancy}
\fancyhead[RO,RE]{\thepage}
\fancyfoot[CO,CE]{}
%Termina configuracion

\chapter*{Summary}
\addcontentsline{toc}{chapter}{Resumen}
\setstretch{1.5} %Regresa el interlineado a 1.5


\normalsize
\noindent The rise of IoT will lead to an explosion in the volume of data
collected, transmitted and processed. This will require novel and optimized
solutions. Besides that, the power consumption of the IoT’s servers is a key
part to considerate. If current trends continue, a petaflop system will require
100 megawatts. To address this problem the trend is towards the autonomous and
responsible behavior of resources Victor will show a network of
ultra-low-voltage Intel microprocessors platforms processing their own workload
without the need of an external HPC system. This is possible due to
configurations in the service manager, kernel and the adaptation of distributed
communication protocols (MPI) 
\clearpage
