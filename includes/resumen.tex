%Empieza configuracion
\setstretch{1.0}
\titleformat{\chapter}{\Huge\bfseries}{\thechapter}{0 pt}{\rule{340 pt}{3 pt}\\}
\titlespacing{\chapter}{100 pt}{-25 pt}{40 pt}[10 pt]	
\pagestyle{fancy}
\fancyhead[RO,RE]{\thepage}
\fancyfoot[CO,CE]{}
%Termina configuracion

\chapter*{Summary}
\addcontentsline{toc}{chapter}{Summary}
\setstretch{1.5} %Regresa el interlineado a 1.5


\normalsize
\noindent 

The use of IoT devices is rising quickly, by 2022 is it expected to have 14
billions of IoT devices creating a rise of data processed, transmitted and
stored. It would create so much traffic that would be similar to a security
attack. Apart from the problems of transmitted data, power consumption of
computing systems that store and process data generated by IoT devices is
important to be studied. If current trends continue, a system to handle IoT
data would require 100 megawatts. This research work proposes a method to
improve the use of IoT platforms by reducing the amount of data
transmitted, as well as, the amount of high performance servers needed for data
processing. At the end a detailed case of study based on industrial benchmarks
for distributed systems, power consumption analysis and optimal operating
system selection, describes a methodology to maximize power efficiency.

\clearpage
