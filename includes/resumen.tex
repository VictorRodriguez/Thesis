%Empieza configuracion
\setstretch{1.0}
\titleformat{\chapter}{\Huge\bfseries}{\thechapter}{0 pt}{\rule{340 pt}{3 pt}\\}
\titlespacing{\chapter}{100 pt}{-25 pt}{40 pt}[10 pt]	
\pagestyle{fancy}
\fancyhead[RO,RE]{\thepage}
\fancyfoot[CO,CE]{}
%Termina configuracion

\chapter*{Summary}
\addcontentsline{toc}{chapter}{Summary}
\setstretch{1.5} %Regresa el interlineado a 1.5


\normalsize
\noindent 

The use of IoT devices is rising quickly, by 2022 it is expected to have 14
billions of IoT devices; which will create a rise on processed, transmitted and
stored data.  They would create so much Internet traffic that it would be
compared to a security attack. Besides the problems of transmitted data, power
consumption of systems that store and process data generated by IoT devices is
important to be studied. If current trends continue, it is estimated that a
system that would handle IoT data would require 100 megawatts. This research
work studies a cluster of embedded boards, which form an IoT platform, to
explore reduction of transmitted data to high performance servers needed for
remote data processing. It also compares power efficiency of the embedded
cluster to a desktop (or traditional) computing system. MPI benchmarks are
utilized to quantify power consumption and performance.  Results are compared
among multiple operating systems to select the most power efficient. This work
provides a way to determine whether applications can take advantage of the
communication between IoT devices to process their own data instead of sending
information to data centers. As a result of the experiments is proved that in
some cases, the power efficiency of a cluster of embedded systems is higher
than a desktop (or traditional) computing system.

\clearpage
