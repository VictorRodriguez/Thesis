%Empieza configuracion
\setstretch{1.0}
\titleformat{\chapter}{\Huge\bfseries}{\thechapter}{0 pt}{\rule{340 pt}{3 pt}\\}
\titlespacing{\chapter}{100 pt}{-25 pt}{40 pt}[10 pt]	
\pagestyle{fancy}
\fancyhead[RO,RE]{\thepage}
\fancyfoot[CO,CE]{}
%Termina configuracion

\chapter*{Resumen}
\addcontentsline{toc}{chapter}{Resumen}
\setstretch{1.5} %Regresa el interlineado a 1.5


\normalsize
\noindent El resumen es una s'intesis de la tesis. Generalmente incluir'a la definici'on del problema, el procedimiento o m'etodos, los resultados y las conclusiones. Esta secci'on deber'a tener un m'aximo de trescientas cincuenta palabras incluyendo preposiciones. Las palabras en el t'itulo no se cuentan como parte del resumen. El resumen debe escribirse con claridad, ya que 'esta es la referencia que se hace p'ublica inmediatamente en los servicios electr'onicos de b'usqueda de informaci'on. Deber'a escribirse a doble espacio. No se recomienda usar diagramas ni f'ormulas en esta secci'on.

En este documento se presenta la estructura de la propuesta de tesis que los alumnos del curso Seminario de Innovaci'on y Creatividad deben presentar para acreditar la materia. Se describen los elementos de la propuesta y el formato que dichos elementos deben llevar. En la portada debe ir el t'itulo tentativo de la tesis, el nombre del autor, la instituci'on de educaci'on superior en la que se realiza el trabajo de tesis, el mes y a'no en que se entreg'o la propuesta. Despu'es de la portada va una p'agina de aprobaci'on que debe ser firmada por el director de la tesis y los sinodales una vez el proyecto haya sido aprobado. En el caso que el documento sea una propuesta de tesis, se debe llenar el formato de Registro de Tesis que se debe entregar a Servicios Escolares antes de que termine el seminario. Este Registro es una condici'on necesaria para que el alumno pueda inscribirse a la materia que sigue: Tesis I.

Posteriormente, puede seguir una dedicatoria que es de car'acter opcional y una secci'on de reconocimientos en la que deben mencionarse aquellas instituciones o personas, si las hay, que est'en proporcionando ayuda o apoyo financiero al proyecto. Por 'ultimo sigue el resumen de la tesis, cuya extensi'on debe ser de una p'agina a dos p'aginas m'aximo.

El contenido muestra las diferentes secciones y cap'itulos de la tesis. Las p'aginas se indican con numeraci'on ar'abiga comenzando desde el cap'itulo 1. Las p'aginas de todas las secciones anteriores al cap'itulo 1 se identifican con numeraci'on romana. Si se usaron tablas o figuras en la tesis, deben incluirse las listas correspondientes de tablas o figuras despu'es de la secci'on de contenido.

Los cap'itulos de introducci'on, desarrollo y conclusiones vienen despu'es, numerados a partir del n'umero 1, con las referencias de cada cap'itulo puestas al final del mismo usando el formato del Institute of Electrical and Electronics Engineers (IEEE). Los ap'endices se identifican con letras (Ap'endice A, Ap'endice B, etc.), y tambi'en pueden llevar referencias bibliogr'aficas. Finalmente, puede haber una s'intesis biogr'afica del autor de tesis.
\clearpage